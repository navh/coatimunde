\documentclass[]{report}
\usepackage[utf8]{inputenc}
\usepackage[english]{babel}
\usepackage{fancyhdr}
\usepackage[margin=3cm]{geometry}
\usepackage[super]{nth}

\addto\captionsenglish{\renewcommand{\chaptername}{Part}}

\pagestyle{fancy}
\fancyhf{}
\fancyhead[LE,LO]{DID - 03 SOR (Hebb \& Stephan)}
\fancyfoot[CE,CO]{\leftmark}
\fancyfoot[LE,RO]{\thepage}


% Title Page
\title{Statement of Requirements}
\author{Officer Cadet 27714 Amos Navarre Hebb \\ and \\ Officer Cadet 27555 Kara Stephan}

\usepackage{graphicx}

\begin{document}
	
\begin{titlepage}
	\begin{center}
		\vspace*{1cm}
		
		\LARGE\textsc{Royal Military College of Canada}\normalsize
		
		\vspace{0.2cm}
		
		\textsc{Department of Electrical and Computer Engineering}
		
		\vspace{1.5cm}
		
		\includegraphics[width=0.3\textwidth]{rmcLogo.png}
		
		\vspace{1.5cm}
		
		\LARGE{DID-03 - Statement of Requirements}\normalsize
		
		\vfill
		
		\textbf{Presented by:}\\Amos Navarre \textsc{Hebb} \& Kara \textsc{Stephan}\\
		\vspace{0.8cm}
		\textbf{Presented to:}\\Dr. Sidney \textsc{Givigi}
		\vspace{0.8cm}
		
		\today
		
	\end{center}
\end{titlepage}

% \begin{abstract}
% \end{abstract}

\tableofcontents
\newpage

\chapter{Introduction}

\section{Document Purpose}

The purpose of this document is to outline the project requirements. That is, what the project is to accomplish once all of the requirements outlined in this document have been completed and to what standard they shall be considered done. The benefits of meeting these requirements and solving this problem will be. This document will then identify the constraints that these requirements impose on this project will be.

\subsection{Obstacle Avoidance}

Both in the consumer and professional sectors the use of autonomous aerial vehicles is growing quickly. 
Obstacle avoidance is the task of satisfying a control objective, in this case moving toward a visual target, while subject to non-intersection or non-collision position constraints. Those later constraints are, in this case, to be dynamically created while moving in a reactive manner, instead of being pre-computed.

\subsection{Uncrewed Aerial Vehicles}

Very generally any powered vehicle that uses aerodynamic forces to provide lift that does not carry a human operator can be considered an uncrewed aerial vehicle. Currently most of these vehicles make up a single component of a larger Uncrewed aircraft system. 

An Uncrewed aircraft system (UAS), or Remotely piloted aircraft system (RPAS), is an aircraft without a human pilot on-board, instead controlled from an operator on the ground. 
Such a system can have varying levels of automation, something as simple as a model aircraft has limited capability. An UAS capable of detecting, recognizing, identifying, and tracking targets of interest in complex environments and integrate with the systems required to process and fuse the collected information into actionable intelligence while operating in a low-to-medium threat environment is the current goal of the RPAS project by the RCAF. 

Flying a UAS requires a secure link to the operator off-board. Maintaining this link, particularly while flying close to the ground where more opportunities for interference are introduced, and in environments where potentially hostile actors may be attempting to jam communications, necessitate a level of automation on-board capable of maintaining flight while denied navigation information.

\section{Aim}

The aim of this project is to build an air robot that is able to identify a target and move to the target, avoiding any obstacles that are in the way. 

Tracking targets of interest in complex environments with an uncrewed aerial vehicle is the ultimate goal of this project. 

\subsection{Targets of Interest}

This target will be an object in the environment that has already been identified visually by an operator before navigation information is denied, or pre-programmed into the UAV before operation.

\subsection{Complex Environment}

The proposed use case of this project would be an environment with obstacles that the UAV could potentially collide with. Flying at sufficiently low altitude that trees or buildings could come between the UAV and the target of interest is the core of the project.


\section{Scope}
The scope of the project is limited due to only testing indoors. This limits the range at which we can set up targets from the robot for identification, and it constraints certain characteristics of the obstacles. To fit everything inside for testing we will be using smaller obstacles. 
(I don't know what else he wants, it says clearly state what you will do but isn't that the aim??).
A possible end goal of this project is...  
 

\chapter{Requirement Definition Activities}

\section{Information}

\subsection{Meetings with Dr. Givigi}

\section{How References Were Used}

\subsection{First Reference}
Don't forget most references will be used in the background part. 

\subsection{Another Reference}


\chapter{Product Requirements}

\section{Functional Requirements (FR)}

\subsection{FR-01: State Machine}
Create a state machine for both land and air robots that allows for a movement decision to be made based on an input from the robot. 

\subsection{FR-02: Movement Toward Target}
The air and land robots will be able to move toward a target under control.

\subsection{FR-03: Trajectory Library}
Create a trajectory library that will contain all possible allowed movements for both the land and air robots.

\subsection{FR-04: Trajectory Library Updating State Machine}
The Trajectory Library will update the state machine when a movement was made. The state machine will then account for this displacement and later correct the movement to keep the robot on the path to the target. 

\subsection{FR-05: Identification of Target}
The robot shall recognize optically a goal and be able to give information on the targets location relative to the robot. 

\subsection{FR-06: Avoiding Obstacle}
The robot shall be able to make a deviation from its current movement pattern to avoid an obstacle in its path and then return to this pattern.

\subsection{FR-07: Identification of Obstacle}
The robot shall recognize optically obstacles in its environment and identify where they are relative to itself.

\subsection{FR-is 08: Multiple Obstacles}
The robot shall be able identify multiple obstacles and determine which is the most dangerous and avoid it accordingly. 


\section{Performance Requirements (PR)}

\subsection{PR-01: Target Identification}
The robot will be able to identify and locate OpenCV's ArCuo shapes within a 15m radius. See FR-05.

\subsection{PR-02: State Machine Corrections}
The state machine will be able to correct for movements made and put the robot back on its original path without deviating more than 15cm (we need to pick a acceptable deviation). See FR-04.


\section{Interface Requirements (IR)}

\subsection{IR-01: Turtlebot Communication through ROS}
Communication to the Turtlebot will be done through Robot Operating System through USB.

\subsection{IR-02: Air Robot Communication}
The air robot will be communicating through Robot Operating System over wireless network.


\section{Simulation Requirements (SimR)}

\subsection{SimR-01: Air Robot}
The final product will be an air robot, but for simulation we will be using a TurtleBot to create and test our systems. 


\section{Implementation Requirements (ImpR)}

\subsection{ImpR-01: Turtlebot Robot Operating System}
The simplest obstacle avoidance algorithm must be implemented on a Turtlebot using the Robot Operating System.


\section{Schedule Restrictions (SR)}

\subsection{SchR-01: First Prototype}

The first functional prototype shall be available for Beta testing no later than November \nth{1}. 

\chapter{Risk Assessment}

\section{Risks}

\section{Likelihood}

\section{Impact}

\chapter{Conclusion}

\section{Summary}

\section{Link to Preliminary Design Specification}

\chapter*{References}

\end{document}          
