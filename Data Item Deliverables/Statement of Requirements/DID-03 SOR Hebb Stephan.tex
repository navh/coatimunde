\documentclass[]{report}
\usepackage[utf8]{inputenc}
\usepackage[english]{babel}
\usepackage{fancyhdr}
\usepackage[margin=3cm]{geometry}
\usepackage[super]{nth}

\addto\captionsenglish{\renewcommand{\chaptername}{Part}}

\pagestyle{fancy}
\fancyhf{}
\fancyhead[LE,LO]{DID - 03 SOR (Hebb \& Stephan)}
\fancyfoot[CE,CO]{\leftmark}
\fancyfoot[LE,RO]{\thepage}


% Title Page
\title{Statement of Requirements}
\author{Officer Cadet 27714 Amos Navarre Hebb \\ and \\ Officer Cadet 27555 Kara Stephan}

\usepackage{graphicx}

\begin{document}
	
\begin{titlepage}
	\begin{center}
		\vspace*{1cm}
		
		\LARGE\textsc{Royal Military College of Canada}\normalsize
		
		\vspace{0.2cm}
		
		\textsc{Department of Electrical and Computer Engineering}
		
		\vspace{1.5cm}
		
		\includegraphics[width=0.3\textwidth]{rmcLogo.png}
		
		\vspace{1.5cm}
		
		\LARGE{DID-03 - Statement of Requirements}\normalsize
		
		\vfill
		
		\textbf{Presented by:}\\Amos Navarre \textsc{Hebb} \& Kara \textsc{Stephan}\\
		\vspace{0.8cm}
		\textbf{Presented to:}\\Dr. Sidney \textsc{Givigi}
		\vspace{0.8cm}
		
		\today
		
	\end{center}
\end{titlepage}

% \begin{abstract}
% \end{abstract}

\tableofcontents
\newpage

\chapter{Introduction}

\section{Document Purpose}

The purpose of this document is to outline the project requirements. That is, what the project is to accomplish once all of the requirements outlined in this document have been completed and to what standard they shall be considered done. The benefits of meeting these requirements and solving this problem will be. This document will then identify the constraints that these requirements impose on this project will be.

\section{Background}

Both in the consumer and professional sectors the use of autonomous aerial vehicles is growing quickly. 

Background is going to be a long section and maybe we should have subsections

\section{Aim}

\section{Scope}

\chapter{Requirement Definition Activities}

\section{Information}

\subsection{Meetings with Dr. Givigi}

\section{How References Were Used}

\subsection{First Reference}

\subsection{Another Reference}

\chapter{Product Requirements}

\section{Functional Requirements (FR)}

\subsection{FR-01: Movement Toward Target}

Be able to control the robot using ROS. In particular, given that our task is to move toward a target the robot must be able to do so in a manner that is controlled.

\subsection{FR-02: Identification of Target}

The robot shall recognize optically a goal and be able to give information on the targets location relative to the robot.

\subsection{FR-03: Avoiding Obstacle}

The robot shall be able to make a deviation from its current movement pattern to avoid an obstacle in its path and then return to this pattern.

\subsection{FR-04: Identification of Obstacle}

The robot shall recognize optically obstacles in its environment and identify where they are relative to itself.

\section{Performance Requirements (PR)}

\subsection{PR-01}

\section{Interface Requirements (IR)}

\subsection{IR-01}

\section{Simulation Requirements (SimR)}

\subsection{SimR-01}

\section{Implementation Requirements (ImpR)}

\subsection{ImpR-01}

\subsection{ImpR-02: Turtlebot Robot Operating System}

The simplest obstacle avoidance algorithm must be implemented on a Turtlebot using the Robot Operating System

\section{Schedule Restrictions (SR)}

\subsection{SchR-01: First Prototype}

The first functional prototype shall be available for Beta testing no later than November \nth{1}

\chapter{Risk Assessment}

\section{Risks}

\section{Likelihood}

\section{Impact}

\chapter{Conclusion}

\section{Summary}

\section{Link to Preliminary Design Specification}

\chapter*{References}

\end{document}          
